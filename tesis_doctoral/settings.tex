%paquetes

\usepackage{graphicx}
\usepackage[spanish]{babel} % option es-lcroman para numeros romanos en minuscula
\usepackage[utf8]{inputenc}
\usepackage{textcomp}
\usepackage{float}
\usepackage[labelfont=bf, textfont=normal,
			justification=justified,
			singlelinecheck=false]{caption}
\usepackage[labelfont=bf, textfont=normal,
			justification=justified,
			singlelinecheck=false]{subcaption}
\usepackage{multirow}
\usepackage{amsmath}
%\usepackage{subfig}
%\usepackage[retainorgcmds]{IEEEtrantools} %% Esto es para insertar links a páginas web.
\usepackage{multicol}
\usepackage[table]{xcolor} %para pintar celdas con el comando \cellcolor{red}
\usepackage{hyperref} % para poner liks dentro y fuera del documento con el comando \href{<my_url>}{<description>}. Buscar mas funcionalidades.
\usepackage[noabbrev, nameinlink, spanish]{cleveref} % para que agregue tipo a la referencia (figura, ec, etc)

\hypersetup{colorlinks, citecolor=black,
   		 	filecolor=black, linkcolor=black,
    		urlcolor=black}
\usepackage[version=4]{mhchem} % Para graficar reacciones químicas

\usepackage{seqsplit} % package for splitting long strings of characters link, DNA, mRNA, Proteins, numbers, etc.

%agregar notas al pdf
\usepackage{todonotes}
\presetkeys{todonotes}{inline}{}

% Lenguajes de Programacion
%\usepackage[]{mcode} % paquete para agregar código de MatLab. Es necesario copiar el mcode.sty https://www.sharelatex.com/project/5539017293616a9c19e2cd24
%\usepackage{listings} %para incluir varios lenguajes de programacion (Ej: Python)

\usepackage{natbib} % take out square brackets from apalike citation

% Caracteristicas de paginas

\usepackage[top=3cm, bottom=3cm,
			inner=2cm, outer=1.5cm]{geometry}			% Page margin lengths

%\pdfpagewidth 8.5in
%\pdfpageheight 11in
%\setlength\oddsidemargin{-0,21in}
%\setlength\evensidemargin{-0,21in}
%\setlength\topmargin{-2cm}
%\setlength\textwidth{7in}
%\setlength\textheight{23.7cm}
\setlength\parskip{0.1in}


% Caracteristicas de estilo

%\usepackage{setspace}
%\onehalfspacing

% Cambiar nombre de tablas

\renewcommand{\listtablename}{Indice de tablas}
	\renewcommand{\tablename}{Tabla}


% Formato de los títulos de capítulos
\usepackage{titlesec}		
\titleformat{\chapter}[display]
  {\Huge\bfseries\filcenter}
  {{\fontsize{50pt}{1em}\vspace{-4.2ex}\selectfont \textnormal{\thechapter}}}{1ex}{}[]
  
% Paquete para agregar apéndices
\usepackage[header, title, titletoc]{appendix}
%\renewcommand{\appendixname}{Apéndice}


% Cabeza y pie de página (Select TWOSIDE or ONESIDE layout below)
\usepackage{fancyhdr}								
\pagestyle{fancy}
\setlength{\headheight}{15pt}
\renewcommand{\chaptermark}[1]{\markboth{\thechapter.\space#1}{}} 


% Select one-sided (1) or two-sided (2) page numbering
\def\layout{2}	% Choose 1 for one-sided or 2 for two-sided layout
% Conditional expression based on the layout choice
\ifnum\layout=2	% Two-sided
    \fancyhf{}			 						
	\fancyhead[LE,RO]{\nouppercase{ \leftmark}}
	\fancyfoot[LE,RO]{\thepage}
	\fancypagestyle{plain}{			% Redefine the plain page style
	\fancyhf{}
	\renewcommand{\headrulewidth}{0pt} 		
	\fancyfoot[LE,RO]{\thepage}}	
\else			% One-sided  	
  	\fancyhf{}					
	\fancyhead[C]{\nouppercase{ \leftmark}}
	\fancyfoot[C]{\thepage}
\fi


% Defino Funciones comunes

\newcommand{\ening}[1]{\textit{#1}}