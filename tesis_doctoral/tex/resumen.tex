

\begin{center}
    {\bf \Large  Microespectroscopía aplicada a la cuantificación y modelado de la propagación de señales en redes biológicas\\}
    
    \vspace{4cm}
    
    % 200 palabras
    
    \textbf{Resumen}
    
    Las redes de señalización biológicas presentan módulos complejos e interconectados que producen una plétora de respuestas posibles a diversos estímulos. Debido a la variabilidad intrínseca a estos sistemas, se torna necesario multiplexar el estado de varios nodos simultáneamente para comprender su dinámica e interacción. En esta tesis se eligió estudiar la cascada apoptótica como sistema modelo de este tipo de redes. La apoptosis es un proceso de muerte celular programada crucial en organismos multicelulares y cuya disfunción puede resultar en el desarrollo de cáncer, entre otras patologías. Considerando que dicha cascada tiene una elevada variabilidad en su inicio, del orden de horas, mientras que el tiempo de activación entre sus nodos esta mejor conservado, del orden de minutos, resulta necesario implementar técnicas correlativas y resueltas en el tiempo. Con el objetivo de generar un mejor entendimiento de dicha red, se la modeló utilizando ecuaciones diferenciales ordinarias para describir su comportamiento ante ligandos extracelulares así como estrés intracelular. Se modificaron los biosensores utilizados para controlar mejor la perturbación introducida al sistema. Finalmente, la sinergia entre experimentos, análisis de datos y modelado que hizo posible el diseño, de forma constructiva, de un único modelo integrado capaz de predecir resultados experimentales propios y ajenos podría extrapolarse al estudio de otras redes.
    % 208 palabras
    \vspace{2cm}
    
    
\end{center}

% no menos de 5 palabras o frases

\textbf{Palabras clave}: Espectroscopia de correlación, FRET, microscopía de fluorescencia, redes biológicas, modelado matemático

\cleardoublepage


\begin{center}
    {\bf \Large Microspectroscopy applied to quantification and modelling of signal propagation in biological networks\\}
    
    \vspace{4cm}
    
    % 200 palabras
    
    \textbf{Abstract}
    
    Biological signalling networks are composed of complex and interconnected modules capable of producing a plethora of responses to different stimuli. Due to their intrinsic variability, it is necessary to simultaneously multiplex several nodes to understand their dynamics and interaction. Throughout this thesis we chose to study the apoptotic cascade as a model system of this kind of networks. Apoptosis is a programmed cell death mechanism crucial in multicellular organisms, whose disfunction may result in cancer, amongst other pathologies. Considering its high onset variability, order of hours, while activation time between nodes is better conserved, order of minutes, it is necessary to use correlative and time resolved techniques. Aiming to gain better understanding of such network, an ordinary differential equations based model was implemented to describe its behaviour against extracellular ligands as well as intracellular stress. Biosensors were modified to better control for their perturbation introduced to the system. Finally, the synergy between experiments, data analysis and modelling that made possible the design, in a constructive manner, of a single integrative model capable of predicting our own and others experimental results could be extrapolated to other networks.
    % 185 words
    \vspace{2cm}
    
    
\end{center}

% no menos de 5 palabras o frases

\textbf{Keywords}: Correlation spectroscopy, FRET, Fluorescence microscopy, biological networks, mathematical modelling