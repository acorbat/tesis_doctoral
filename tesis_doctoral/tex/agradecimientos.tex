\chapter*{Agradecimientos}


En primer lugar, agradezco a Dios por haberme dado el espíritu y profundo deseo para estudiar la compleja maquinaria que es la vida, así como le agradezco el haberla creado con su clara impronta, sin la cual siquiera hubiera podido comenzar a estudiarla.

Quiero agradecer a toda mi familia por haberme enseñado y guiado desde temprano en mi vida, especialmente por apoyarme y darme más de lo que necesitaba para cumplir mis metas. Sin ella tendría una identidad completamente distinta y no sería quien soy. Espero poder trasmitir la alegría y apoyo en los que me críe a los demás miembros de mi familia, en especial los más nuevos.

Estoy profundamente agradecido con todos mis amigos que día a día comparten sus experiencias, intercambiamos palabras de aliento y nos aconsejamos mutuamente. También les tengo que agradecer que siempre están para relajarnos y divertirnos un tiempo, algo que es esencial para despejar la cabeza y darle una mano a la creatividad.

A Hernán, quien no solo me dirigió durante tantos años, sino que también me acompaño e hizo de mentor en muchas de las formas de pensar y relacionarme en el ámbito científico. Muchos de los ideales y pasiones que compartimos hoy en día no los tendría si no fuese por él.

Agradezco a todos los miembros del Laboratorio de Electrónica Cuántica que siempre tienen muy buena predisposición para ayudar y están atentos a esos bajones de ánimo (más usuales en ciencia de lo que uno cree) para encarrilar de nuevo las energías. Estoy seguro que no lo dije lo suficiente, pero un excelente ambiente de trabajo es indispensable para la ciencia ya que sino es muy fácil resignarse ante las incesantes dificultades en investigación. Quiero darle un agradecimiento especial a Andrea que además me ayudó en mi formación como docente y me proveyó algunos materiales esenciales para mi trabajo como doctorando y docente. También quiero agradecer Martín Habif por tantas horas de cultivo y microscopía.

Quiero destacar y agradecer al Departamento de Física que día a día desde hace tantos años empujan para adelante haciendo del lugar algo maravilloso. Cada una de las personas con las que tuve la suerte de interactuar me enseñaron que uno siempre se tiene que proponer buscar esos compañeros con los que llegar más lejos.

Agradezco a la Universidad de Buenos Aires por haberme formado y brindado lugar de trabajo. A CONICET por su apoyo en el proyecto de doctorado propuesto y becado. A la Agencia Nacional de Promoción Científica y Tecnológica por los subsidios brindados para algunos de los materiales utilizados. Al Max Planck Group por los subsidios y la oportunidad de realizar muchos de los experimentos que necesité.

A todos aquellos que conocí al pasar por el Instituto Max Planck de Fisiología Molecular y fueron tan amables conmigo. En especial quiero agradecer a Philippe Bastiaens con quien tuve la oportunidad de entablar discusiones muy interesantes, Sven Müller con quien nos divertimos adaptando microscopios, Klaus Schuermann con quien compartimos un gran trabajo y me proveyó parte de los datos y Sarah Imtiaz con quien pude trabajar en colaboración.

A los miembros del grupo de Graciela Boccaccio en el Instituto Leloir con quienes trabajé en colaboración durante varios años y me acompañaron en mi formación como científico en la interdisciplina.

Una mención especial para Alexandra Elbakyan, creadora de Sci-Hub, que me permitió acceder gratuitamente a una infinidad de trabajos para utilizar como bibliografía.