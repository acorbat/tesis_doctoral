\begin{appendices}


%\appendix
\chapter{Repositorios de Código}
\label{repositorio}

\begin{itemize}
    

\item \href{https://github.com/acorbat/img_manager}{\textbf{img\textunderscore manager}}: Consiste en un repositorio que contiene clases y funciones que permiten cargar imágenes adquiridas con el microscopio y la metadata correspondiente, así como una librería de algoritmos que aplican diversas correcciones a las imágenes.

(\href{https://github.com/acorbat/img_manager}{https:\textbackslash \textbackslash github.com\textbackslash acorbat\textbackslash img\textunderscore manager})

\item \href{https://github.com/maurosilber/cellment/tree/master/cellment}{\textbf{Cellment}}: Un paquete desarrollado en conjunto con Mauro Silberberg que permite determinar de forma robusta la distribución de intensidad del fondo, seguir células a lo largo de una secuencia de imágenes, entre otras cosas.

(\href{https://github.com/maurosilber/cellment/tree/master/cellment}{https:\textbackslash \textbackslash github.com\textbackslash maurosilber\textbackslash cellment\textbackslash tree\textbackslash master\textbackslash cellment})

\item \href{https://github.com/acorbat/apoptoside}{\textbf{apoptoside}}: Paquete utilizado para analizar los datos correspondientes a los experimentos de anisotropía y las simulaciones análogas.

(\href{https://github.com/acorbat/apoptoside}{https:\textbackslash \textbackslash github.com\textbackslash acorbat\textbackslash apoptoside})

\item \href{https://github.com/acorbat/caspase_model/tree/master}{\textbf{caspase\textunderscore model}}: Este paquete contiene la implementación en Python del modelo publicado en \cite{Corbat2018} y \textit{Apoptotic Reaction Model} \citep{Corbat2021}.

(\href{https://github.com/acorbat/caspase_model/tree/master}{https:\textbackslash \textbackslash github.com\textbackslash acorbat\textbackslash caspase\textunderscore model\textbackslash tree\textbackslash master})

\item \href{https://github.com/acorbat/anisotropy_errors/tree/master/anisotropy_errors}{\textbf{anisotropy\textunderscore errors}}: Contiene un IPython notebook (que puede correrse online en Binder) que permite simular curvas de fracción de monómeros, anisotropía y células, así como las imágenes adquiridas para estimar como es la propagación de errores durante el análisis.

(\href{https://github.com/acorbat/anisotropy_errors/tree/master/anisotropy_errors}{https:\textbackslash \textbackslash github.com\textbackslash acorbat\textbackslash anisotropy\textunderscore errors\textbackslash tree\textbackslash master\textbackslash anisotropy\textunderscore errors})

\item \href{https://www.ebi.ac.uk/biomodels/MODEL2105210001}{\textbf{MODEL2105210001}}: En Biomodels se encuentra \textit{Apoptotic Reaction Model} en formato SBML para descargar e implementar \citep{Malik-Sheriff2019}.

(\href{https://www.ebi.ac.uk/biomodels/MODEL2105210001}{https:\textbackslash \textbackslash www.ebi.ac.uk\textbackslash biomodels\textbackslash MODEL2105210001})

\end{itemize}

\end{appendices}