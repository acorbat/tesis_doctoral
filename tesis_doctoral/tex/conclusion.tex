\chapter{Conclusiones y Discusiones}

En el presente trabajo, se generó el primer modelo integrador de la cascada apoptótica, que hemos denominado \textit{Apoptotic Reaction Model}, capaz de reproducir la dinámica de sus módulos ante estímulos categóricamente distintos, así como resultados publicados previamente por otros grupos. Para ello, fue necesario desarrollar y caracterizar un arreglo de biosensores que permita observar la actividad de cada módulo de la red de caspasas a nivel de célula única. Con ellos se obtuvo un cuerpo de datos cohesivo que surgió de interrogar la dinámica de cada módulo ante estímulos intrínsecos y extrínsecos. Luego de encontrar discrepancias entre las predicciones de los modelos utilizados y las observaciones experimentales, se identificó una retroalimentación ausente en los modelos, cuyas reacciones debieron ser añadidas para generar un modelo completo. Adicionalmente, se simularon experimentos descriptos en trabajos previos para corroborar que el modelo generado conserva la capacidad predictiva de otros experimentos.

El procedimiento aplicado aquí para estudiar la red de caspasas puede ser extrapolado a otras redes biológicas. A grandes rasgos, éste consistió en diseñar un biosensor y su correspondiente análisis para obtener un observable robusto que sirva para describir la dinámica de los módulos que componen a la red. Luego, la red deberá ser interrogada y perturbada de formas distintas para generar un cuerpo de datos que describa su comportamiento en cada situación. Finalmente, estudiar e integrar diversos modelos que describan cada módulo o la totalidad de la red y corroborar que las modificaciones introducidas describen las nuevas observaciones, así como los experimentos previos.

En el \cref{cap:MatMet}, se planteó un experimento en el que se deseaba estudiar la actividad de una proteasa a partir de un biosensor basado en homoFRET. Combinando modelado y ley de acción de masas con el entendimiento del origen de la señal fotofísica a partir del estado del ensamble de biosensores, se demostró un método para reportar la actividad enzimática a cada tiempo.  Utilizar las curvas de actividad en lugar de la señal del reportero sin procesar trae la ventaja, no solo de su valor biológico, sino que también es posible compararlas con resultados de simulaciones del sistema.

Estimar un observable de índole biológica es importante para comprender el comportamiento de la red, pero a esto hay que agregarle una forma de lidiar con la elevada variabilidad que tienen estos sistemas. Aquí se combinaron dos metodologías que permiten sortear dicho problema: utilizar técnicas de alto rendimiento y un observable robusto. En primer lugar, se buscó automatizar la adquisición y análisis de imágenes para así generar un cuerpo de datos con el que se pueda describir la población de células. Se realizó un estudio detallado de distintos algoritmos que podían ser utilizados, así como posibles sesgos que estos introducían, para generar un flujo de análisis automatizado. En segundo lugar, se comprobó que la diferencia de tiempos de máxima actividad de caspasas esta mucho mejor conservado que utilizar uno sólo de ellos ya que el momento en que se desencadena la apoptosis es mucho más variable que el tiempo que transcurre entre el máximo de actividad de cada caspasa. Se debe destacar también que combinar técnicas de alto rendimiento con un análisis de reducción de datos permite generar cuerpos de datos fácilmente interpretables.

Habiendo discutido la metodología por la cual se interrogaría la red de caspasas de forma automática y cuál es el observable a obtener, se introdujo en el \cref{cap:Biosensores} una caracterización de los biosensores a implementarse. El análisis inicial de las combinaciones de fluoróforos realizado en el Instituto Max Planck de Fisiología Molecular sirvió de base para implementar los sensores basados en homoFRET que se utilizaron en los distintos experimentos. Con el objetivo de traducir las curvas de anisotropía en la actividad correspondiente a cada caspasa, se procedió a caracterizar distintos parámetros de los pares de fluoróforos. Por un lado, se determinó el valor de $b$, la relación entre los brillos de los biosensores en estado dimérico y monomérico, a través de un experimento en el que se generaron tres biosensores de colores distintos pero sensibles a la misma caspasa. Por otro lado, las anisotropías del dímero y monómero se determinaron para cada curva particular debido al efecto que tienen sobre estos valores los errores en la estimación de la intensidad.

Se buscó disminuir la variabilidad en la estequiometría de los biosensores añadidos de forma exógena, que a su vez aumentaba la varianza de los observables biológicos por efecto de secuestro que los biosensores tienen sobre las caspasas, implementando la construcción de un único plásmido que codifique para los tres biosensores separados por secuencias virales 2A. Mediante citometría de flujo y espectroscopia de correlación de fluorescencia se caracterizó la equimolaridad en la expresión de CASPAM, el plásmido desarrollado. Por último, se vio que el desvío estándar de las diferencias de tiempo de máxima actividad de las caspasas es menor en células transfectadas con CASPAM que con los tres sensores en plásmidos separados.

Una vez diseñada la metodología experimental por medio de la cual se interrogará la actividad de caspasas, se procedió a construir un cuerpo de datos que describa la dinámica de la red de caspasas ante estímulos intrínsecos y extrínsecos. Se apreció que la caspasa efectora es la primera en alcanzar el máximo de actividad sin importar el estímulo utilizado. Esto resulta de interés ya que si miramos el flujo de información, uno esperaría que las caspasas iniciadoras sean las primeras en alcanzar el máximo de actividad. Sin embargo, debido a que las caspasas iniciadoras se encuentran inhibidas para evitar que pequeños estímulos desencadenen la cascada, éstas alcanzan su máximo de actividad una vez que las caspasas efectoras las retroalimentan.

Finalmente, se construyó un modelo basado en ecuaciones diferenciales ordinarias y ley de acción de masas para describir la dinámica de la cascada apoptótica ante ambos tipos de estímulo. Como punto de partida, se utilizó un modelo preparado por \cite{Albeck2008} para describir el comportamiento de la red ante estímulos extrínsecos. Modificando las concentraciones iniciales del receptor, el ligando y XIAP fue posible recuperar la dinámica observada ante estímulos extrínsecos sin afectar predicciones previas. No obstante, no fue posible encontrar una combinación de parámetros, en este modelo y en otros similares publicados, que sirva para describir al mismo tiempo la dinámica con estímulos intrínsecos. La diferencia principal entre las predicciones de dichos modelos y lo que se observó experimentalmente, fue el orden de los máximos de actividad de la caspasa intrínseca y la efectora. De esta forma, se detectó que la retroalimentación de la caspasa efectora a la intrínseca estaba ausente en los modelos utilizados. Una vez añadida, se encontró un conjunto de parámetros que permitió describir la dinámica de la red de caspasas ante ambos tipos de estímulos simultáneamente.

Con el objetivo de corroborar que el modelo desarrollado conserva la capacidad predictiva, se simularon experimentos previos efectuados por otros grupos. En primer lugar, se simuló la red con intensidades crecientes de estímulo para ver que el tiempo en el que se desencadena la apoptosis, representado aquí por el tiempo de máxima actividad de la caspasa efectora, es inversamente proporcional a la intensidad del estímulo. En segundo lugar, se vio que el modelo conserva el desencadenamiento rápido de la cascada ya que ante la formación de pocos poros en la membrana mitocondrial, su contenido se libera rápidamente al citosol. Luego, se realizó un estudio de sensibilidad de observables ante cambios de la concentración inicial de todas las especies distintas de cero. Se pudo apreciar que muchas de las especies cuyo cambio afectaba el tiempo en el que se desencadenaba la cascada también había sido reportado en experimentos previamente. Por último, en forma análoga a experimentos donde se \textit{knockea} algún subconjunto de caspasas, se simularon casos en donde algunas caspasas estaban ausentes. Esto sirvió para estudiar el flujo de la señal en la red perturbada y se pudieron reproducir los resultados de los experimentos. Más importante, se pudieron reproducir resultados de experimentos donde se bloquean las caspasas efectoras que no se podrían reproducir con la topología de los modelos anteriores.

Unificar la información adquirida a través de los distintos experimentos en los que se pone a prueba la cascada apoptótica en un único modelo integrador es importante para generar predicciones confiables en terapias personalizadas, como en \cite{Eduati2020}. Como discutieron previamente \cite{Santolini2018}, determinar la topología de la red es crucial para descubrir los comportamientos emergentes del sistema bajo estudio. Modelos con un elevado nivel de detalle y que incluyen muchas especies podrían integrarse con otros modelos que describan otras vías, llevando en última instancia a uno de los desafíos propuestos por \cite{Karr2015}, un modelo de célula completa.

Con el objetivo de que tanto la metodología utilizada, así como los sensores desarrollados y los modelos diseñados puedan ser aplicados en distintos contextos y mejorados, todo fue publicado y subido en artículos científicos y repositorios. Los modelos desarrollados se encuentran disponibles para descargar en \href{https://github.com/acorbat/caspase_model/tree/master}{caspase\textunderscore model} y \href{https://www.ebi.ac.uk/biomodels/MODEL2105210001}{Biomodels} (MODEL2105210001, ver apéndice \ref{repositorio}). Tener fácil acceso a los modelos desarrollados permite probarlos en contextos distintos, y de ser necesario, completarlos. 

En esta tesis, se interrogó la dinámica de la cascada apoptótica ante ciertos estímulos extrínsecos e intrínsecos en células inmortalizadas de cáncer de cérvix generándose un modelo integrador que describe la dinámica de las caspasas y recapitula resultados de experimentos previos. Dentro de la amplia variedad de contextos biológicos donde apoptosis juega un papel importante, aquí nos hemos focalizado en estos dos estímulos para desarrollar una metodología integral de cuantificación y modelado de una red. La metodología desarrollada es una herramienta valiosa para analizar como se reconfigura la red en otras situaciones. Sería interesante introducir la síntesis y degradación de especies del modelo que no fueron tenidas en cuenta ya que en nuestro caso particular todos los experimentos se realizaron con cicloheximida por lo que la síntesis proteica estaba inhibida. Incluir síntesis y degradación de especies en el modelo será importante para predecir comportamientos ante estímulos oscilatorios o actividad subletal de caspasas, como durante la plasticidad neuronal \citep{Unsain2015}. Por otro lado, este modelo podría utilizarse para comparar la propagación de señales por la red en apoptosis tipo I (independiente de permeabilización mitocondrial) o tipo II (dependiente de permeabilización mitocondrial) en función de la concentración de XIAP y procaspasa \citep{Aldridge2014}. También resulta de interés analizar el papel de la cascada apoptótica en el control de la homeostasis de tejidos epiteliales, como fue analizado por \cite{Gagliardi2021}, \cite{Aoki2013} y \cite{Albeck2013}, así como su rol en la formación de patrones en el desarrollo epidermal de \textit{Drosophila Melanogaster} \citep{Crossman2018}. Estos son solo una pequeña selección de los diversos contextos en los que se podría estudiar la cascada en búsqueda de una descripción aún más completa.

% La metodología desarrollada es una herramienta valiosa para analizar en un futuro si la red se reconfigura en otras situaciones y queda disponible . 

%.. recapitular en una oracion. Se estudio la respuesta a dos estimulos y seria interesante en un futuro ver... otros tipos de tejido (apoptosis tipo 1 vs tipo 2), contexto, etc. Si la red se reconfigura ante estos contextos. 


% El elevado nivel de detalle que contiene el modelo posibilita la inclusión de comportamientos característicos de la red de caspasas. Aunque es posible simplificar el modelo para reproducir 


% Biological signaling networks are composed of complex and intertwined modules producing a plethora of responses specific to each stimuli. Devising a single model applicable to various scenarios has been limited by a lack of robust observables. Many previous numerical models have been tailored to fit experimental datasets in restricted scenarios, failing to predict response to different stimuli. Apoptosis is a programmed cell death mechanism crucial in multicellular organisms, whose dysfunction may result in cancer, amongst other pathologies. By means of homoFRET based biosensors and polarization anisotropy live imaging microscopy, I constructed a comprehensive dataset in which the activity of three caspases was simultaneously monitored upon intrinsic or extrinsic stimulation. In spite of high onset variability (in the order of hours), mathematical modelling showed that the time-point of maximum activity is a robust proxy for node activation, and that timing between nodes is better conserved (in the order of minutes). As biosensors introduce delays in their sensed nodes, we co-expressed them from a single plasmid and characterized their equimolarity with Fluorescence Activated Cell Sorting and Correlation Spectroscopy to reduce perturbation variance. Surprisingly, effector caspases reached maximum activity first, irrespective of stimuli used, revealing that feedback to initiator caspases is critical, prompting us to create the first integrated Apoptotic Reaction Model. By simulating previously published experiments, we showed that ARM faithfully reproduces them, thus providing a single robust model applicable across different scenarios.